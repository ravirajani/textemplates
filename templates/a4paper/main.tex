\totalpts[100]
\title{Exam Title}
\author{Ravi Rajani}
\date{10 January 2025\\ 09:00 -- 12:00 h}
\begin{document}
\maketitle
\begin{instructions}
    \item You are \textbf{not allowed} to use a calculator.

    \item The total number of points that can be earned is \totalpts. The grade for this exam is determined by dividing the total score by 10.

    \item For each question, write down the \textbf{precise and complete} solutions in the answer box. You will not get any points for an answer without explanation.

    \item At the end of the exam, there is \textbf{extra answer space} in case you didn't manage to fit your complete solutions in the corresponding answer box. Please indicate at the particular question that you continued on the extra space and, within the extra space, please refer to the original question number.

    \item After the exam, the exam itself and the complete solutions will be posted on Canvas. After the grading has finished, you have the opportunity to inspect your graded exam.
\end{instructions}
\begin{question}[8]
    Description.
    \begin{subparts}
        \subpart[4] First part.
        \subpart[1] Second part.
    \end{subparts}
\end{question}
\begin{soltn}
    This is hidden unless option \textsf{solutions} is provided to the \textsf{documentclass} definition.
\end{soltn}
\begin{question}[10]
    \nodescription
    \begin{subparts}
        \subpart[5] First part.
        \subpart[5] Second part.
    \end{subparts}
\end{question}

\end{document}